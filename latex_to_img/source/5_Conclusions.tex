This work unified the analysis of frequency estimation under Local Differential Privacy, identifying Optimized Local Hashing (OLH) as the superior protocol due to its domain-independent variance and low communication cost. In contrast, k-Randomized Response (kRR) proves unsuitable for large domains due to the linear scaling of both variance and vulnerability. Our security evaluation unveiled a critical "privacy-security paradox": stronger privacy guarantees (lower $\epsilon$) necessitate higher noise, which inadvertently amplifies the impact of poisoning attacks by masking malicious inputs. While OUE and OLH offer stable resistance, kRR becomes catastrophically insecure as the domain grows. Future research should focus on server-side robust aggregation techniques to mitigate these attacks, extend the threat model to adaptive adversaries, and analyze the impact of poisoning on complex analytical tasks beyond simple frequency estimation.