\documentclass[11pt,a4paper]{article}
\usepackage[utf8]{inputenc}
\usepackage[margin=1in]{geometry}
\usepackage{hyperref}
\usepackage{booktabs}
\usepackage{array}
\usepackage{longtable}

\hypersetup{
    colorlinks=true,
    linkcolor=blue,
    urlcolor=blue,
    citecolor=blue
}

\title{Third Paper Selection: Comparison of Candidates}
\author{}
\date{\today}

\begin{document}

\maketitle

\section{Selected Papers for Detailed Study}

\subsection{Paper Links}

\begin{longtable}{p{0.45\textwidth}p{0.30\textwidth}p{0.15\textwidth}}
\toprule
\textbf{Paper Title} & \textbf{Link} & \textbf{Shortform} \\
\midrule
\endfirsthead

\multicolumn{3}{c}%
{\tablename\ \thetable\ -- \textit{Continued from previous page}} \\
\toprule
\textbf{Paper Title} & \textbf{Link} & \textbf{Shortform} \\
\midrule
\endhead

\midrule
\multicolumn{3}{r}{\textit{Continued on next page}} \\
\endfoot

\bottomrule
\endlastfoot

Data Poisoning Attacks to Local Differential Privacy Protocols &
\href{../papers/attacks.pdf}{Local PDF} &
Cao2020 \\
\midrule

Locally Differentially Private Protocols for Frequency Estimation &
\href{../papers/LDP_Freq_Est.pdf}{Local PDF} &
Wang2017 \\

\end{longtable}

\subsection{Publication Details}

\begin{longtable}{p{0.12\textwidth}p{0.08\textwidth}p{0.25\textwidth}p{0.08\textwidth}p{0.35\textwidth}}
\toprule
\textbf{Shortform} & \textbf{Year} & \textbf{Venue} & \textbf{Rank} & \textbf{Google Scholar Category} \\
\midrule
\endfirsthead

\multicolumn{5}{c}%
{\tablename\ \thetable\ -- \textit{Continued from previous page}} \\
\toprule
\textbf{Shortform} & \textbf{Year} & \textbf{Venue} & \textbf{Rank} & \textbf{Google Scholar Category} \\
\midrule
\endhead

\midrule
\multicolumn{5}{r}{\textit{Continued on next page}} \\
\endfoot

\bottomrule
\endlastfoot

Cao2020 &
2020 &
USENIX Security &
\#1 &
\href{https://scholar.google.es/citations?view_op=top_venues&hl=en&vq=eng_computersecuritycryptography}{Computer Security \& Cryptography} \\
\midrule

Wang2017 &
2017 &
USENIX Security &
\#1 &
\href{https://scholar.google.es/citations?view_op=top_venues&hl=en&vq=eng_computersecuritycryptography}{Computer Security \& Cryptography} \\

\end{longtable}

\section{Candidate Papers}

\subsection{Paper Links}

\begin{longtable}{p{0.45\textwidth}p{0.30\textwidth}p{0.15\textwidth}}
\toprule
\textbf{Paper Title} & \textbf{Link} & \textbf{Shortform} \\
\midrule
\endfirsthead

\multicolumn{3}{c}%
{\tablename\ \thetable\ -- \textit{Continued from previous page}} \\
\toprule
\textbf{Paper Title} & \textbf{Link} & \textbf{Shortform} \\
\midrule
\endhead

\midrule
\multicolumn{3}{r}{\textit{Continued on next page}} \\
\endfoot

\bottomrule
\endlastfoot

Locally Differentially Private Heavy Hitter Identification &
\href{https://www.semanticscholar.org/paper/Locally-Differentially-Private-Heavy-Hitter-Wang-Li/23feaf87ea8e84e51de8a4824c636a15580826fc}{Semantic Scholar} &
Wang2021 \\
\midrule

Discrete Distribution Estimation under Local Privacy &
\href{https://www.semanticscholar.org/paper/Discrete-Distribution-Estimation-under-Local-Kairouz-Bonawitz/151750b36ca75e5639386cc40ebb5a1090449a02}{Semantic Scholar} &
Kairouz2016 \\
\midrule

Further Study on Frequency Estimation under Local Differential Privacy &
\href{https://www.usenix.org/conference/usenixsecurity25/presentation/fang}{USENIX} &
Fang2025 \\

\end{longtable}

\subsection{Publication Details}

\begin{longtable}{p{0.12\textwidth}p{0.08\textwidth}p{0.25\textwidth}p{0.08\textwidth}p{0.35\textwidth}}
\toprule
\textbf{Shortform} & \textbf{Year} & \textbf{Venue} & \textbf{Rank} & \textbf{Google Scholar Category} \\
\midrule
\endfirsthead

\multicolumn{5}{c}%
{\tablename\ \thetable\ -- \textit{Continued from previous page}} \\
\toprule
\textbf{Shortform} & \textbf{Year} & \textbf{Venue} & \textbf{Rank} & \textbf{Google Scholar Category} \\
\midrule
\endhead

\midrule
\multicolumn{5}{r}{\textit{Continued on next page}} \\
\endfoot

\bottomrule
\endlastfoot

Wang2021 &
2021 &
IEEE TDSC &
\#6 &
\href{https://scholar.google.es/citations?view_op=top_venues&hl=en&vq=eng_computersecuritycryptography}{Computer Security \& Cryptography} \\
\midrule

Kairouz2016 &
2016 &
ICML/PMLR &
\#3 &
\href{https://scholar.google.es/citations?view_op=top_venues&hl=en&vq=eng_artificialintelligence}{Artificial Intelligence} \\
\midrule

Fang2025 &
2025 &
USENIX Security &
\#1 &
\href{https://scholar.google.es/citations?view_op=top_venues&hl=en&vq=eng_computersecuritycryptography}{Computer Security \& Cryptography} \\

\end{longtable}

\section{Unreviewed Candidate Papers}

\subsection{Paper Links}

\begin{longtable}{p{0.45\textwidth}p{0.30\textwidth}p{0.15\textwidth}}
\toprule
\textbf{Paper Title} & \textbf{Link} & \textbf{Shortform} \\
\midrule
\endfirsthead

\multicolumn{3}{c}%
{\tablename\ \thetable\ -- \textit{Continued from previous page}} \\
\toprule
\textbf{Paper Title} & \textbf{Link} & \textbf{Shortform} \\
\midrule
\endhead

\midrule
\multicolumn{3}{r}{\textit{Continued on next page}} \\
\endfoot

\bottomrule
\endlastfoot

XXXXXXXXXX &
XXXXXXXXXX &
XXXXXXXXXX \\
\midrule

XXXXXXXXXX &
XXXXXXXXXX &
XXXXXXXXXX \\
\midrule

XXXXXXXXXX &
XXXXXXXXXX &
XXXXXXXXXX \\

\end{longtable}

\subsection{Publication Details}

\begin{longtable}{p{0.12\textwidth}p{0.08\textwidth}p{0.25\textwidth}p{0.08\textwidth}p{0.35\textwidth}}
\toprule
\textbf{Shortform} & \textbf{Year} & \textbf{Venue} & \textbf{Rank} & \textbf{Google Scholar Category} \\
\midrule
\endfirsthead

\multicolumn{5}{c}%
{\tablename\ \thetable\ -- \textit{Continued from previous page}} \\
\toprule
\textbf{Shortform} & \textbf{Year} & \textbf{Venue} & \textbf{Rank} & \textbf{Google Scholar Category} \\
\midrule
\endhead

\midrule
\multicolumn{5}{r}{\textit{Continued on next page}} \\
\endfoot

\bottomrule
\endlastfoot

XXXXXXXXXX &
XXXXXXXXXX &
XXXXXXXXXX &
XXXXXXXXXX &
XXXXXXXXXX \\
\midrule

XXXXXXXXXX &
XXXXXXXXXX &
XXXXXXXXXX &
XXXXXXXXXX &
XXXXXXXXXX \\
\midrule

XXXXXXXXXX &
XXXXXXXXXX &
XXXXXXXXXX &
XXXXXXXXXX &
XXXXXXXXXX \\

\end{longtable}

\section{Analysis of Wang2021}

\subsection{Relevance to Cao2020}

Introduces the PEM, which is mentioned in Cao2020 in the following sections:

\begin{longtable}{p{0.15\textwidth}p{0.75\textwidth}}
\toprule
\textbf{Cao2020 Section} & \textbf{PEM Discussion} \\
\midrule
\endfirsthead

\multicolumn{2}{c}%
{\tablename\ \thetable\ -- \textit{Continued from previous page}} \\
\toprule
\textbf{Cao2020 Section} & \textbf{PEM Discussion} \\
\midrule
\endhead

\midrule
\multicolumn{2}{r}{\textit{Continued on next page}} \\
\endfoot

\bottomrule
\endlastfoot

Section 2.2 &
Introduces PEM as state-of-the-art heavy hitter protocol with iterative prefix-based mechanism using OLH \\
\midrule

Section 4.2 &
Data poisoning attacks (RPA, RIA, MGA) manipulate bits in each iteration to push attacker-chosen items into top-k \\
\midrule

Section 5.3 &
MGA achieves 100\% attack success with $\sim$5\% fake users on multiple datasets \\
\midrule

Section 6.2 &
Fake user detection via frequent itemset mining at each PEM iteration \\

\end{longtable}

\section{Analysis of Fang2025}

\subsection{Contribution}

\begin{itemize}
\item It's a very recent paper and introduces a latest LDP protocol called RWS
\item To be filled
\item To be filled
\end{itemize}

\subsection{Relevance to Wang2017}

Improves upon OUE and OLH protocols, introducing RUE and RLH. The paper discusses OUE and OLH in the following sections:

\begin{longtable}{p{0.20\textwidth}p{0.30\textwidth}p{0.40\textwidth}}
\toprule
\textbf{Fang2025 Section} & \textbf{Protocol} & \textbf{Discussion} \\
\midrule
\endfirsthead

\multicolumn{3}{c}%
{\tablename\ \thetable\ -- \textit{Continued from previous page}} \\
\toprule
\textbf{Fang2025 Section} & \textbf{Protocol} & \textbf{Discussion} \\
\midrule
\endhead

\midrule
\multicolumn{3}{r}{\textit{Continued on next page}} \\
\endfoot

\bottomrule
\endlastfoot

Section 3.2 &
OUE &
Main definition section for Optimized Unary Encoding \\
\midrule

Section 3.3 &
OLH &
Main definition section for Optimized Local Hashing \\
\midrule

Section 3.5 &
OUE \& OLH &
Summary comparing protocols; states OUE and OLH only achieve optimal MSE for large d \\
\midrule

Section 4 &
OUE \& OLH &
Explains that OUE and OLH were optimized using approximate equations that need improvement \\
\midrule

Section 4.1.1 &
OUE $\rightarrow$ RUE &
Introduces Re-optimized Unary Encoding (RUE) built from OUE \\
\midrule

Section 4.1.2 &
OLH $\rightarrow$ RLH &
Introduces Re-optimized Local Hashing (RLH) built from OLH \\
\midrule

Section 4.1.3 &
OUE vs RUE &
Parameter discussion comparing OUE and RUE optimization approaches \\
\midrule

Section 4.2 &
OLH $\rightarrow$ RLH &
Addresses OLH's slow server-side computation and how RLH solves it \\

\end{longtable}

\section{Analysis of Kairouz2016}

This paper is \textbf{not ideal for selection} as it is published in an Artificial Intelligence venue rather than a cryptography one
\end{document}
